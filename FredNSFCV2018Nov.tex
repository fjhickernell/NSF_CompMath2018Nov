%&LaTeX
%Fred Hickernell's CV 
\documentclass[11 pt]{NSFamsart} 
\usepackage{amsmath,amssymb,longtable,ifthen,hyperref,xspace}
\usepackage[dvipsnames]{xcolor}

\setlength{\textwidth}{6.4in}
\voffset -0.57in 
\setlength{\textheight}{8.9in}
\setlength{\oddsidemargin}{0in}
\setlength{\evensidemargin}{0in}
%\textheight 9in

\newcommand\myshade{85}
\colorlet{mylinkcolor}{violet}
\colorlet{mycitecolor}{Aquamarine}
%\colorlet{mycitecolor}{OliveGreen}
\colorlet{myurlcolor}{YellowOrange}

\hypersetup{
	linkcolor  = mylinkcolor!\myshade!black,
	citecolor  = mycitecolor!\myshade!black,
	urlcolor   = myurlcolor!\myshade!black,
	colorlinks = true,
}


\def\reals{{\mathbb{R}}}
\newcommand{\Matlab}{{\sc Matlab}\xspace}
\providecommand{\HickernellFJ}{Hickernell\xspace}

\thispagestyle{empty}
\pagestyle{empty}

\begin{document}

\centerline{\Large \bf Frederick John Hickernell}

\medskip

\noindent
\begin{tabular}{lllll}
	\multicolumn{5}{l}{{\large\textbf{Professional Preparation}}} \\
	\textbf{Institution} & \textbf{Location} & \textbf{Major(s)} & \textbf{Degree} & \textbf{Year} \\
	Pomona College & Claremont, CA & mathematics \& physics & BA & 1977 \\
	Massachusetts Institute  & Cambridge, MA &  mathematics  &  PhD &1981 \\
	\hspace{3cm} of Technology
\end{tabular}

\bigskip

\noindent
\begin{tabular}{lp{13.5cm}}
	\multicolumn{2}{l}{{\large\textbf{Appointments}}} \\
	2018--present & Vice Provost for Research, Illinois Institute of Technology \\
2017--2018 & Director, Center for Interdisciplinary Scientific Computation, Illinois Institute of 
Technology \\ 
2005--present & Professor, Illinois Institute of Technology \\ 
2005--2017 & Chair of Applied Mathematics, Illinois Institute of Technology \\ 
1995--2005  &  Associate Professor \& Professor of Mathematics, Hong Kong Baptist
University\\ 
1989--2002 & Head of Mathematics, Hong Kong Baptist College/University \\ 
1985--1995 &  Lecturer \& Senior Lecturer of Mathematics, Hong Kong Baptist College \\
1981--1985 & Assistant Professor, University of Southern California
\end{tabular}

\bigskip

\noindent {\large \textbf{Products}} (for citation counts see 
\href{http://scholar.google.com/citations?user=dJbMJG8AAAAJ}{\nolinkurl{scholar.google.com}} 
(Fred J. Hickernell))

\bigskip 

\centerline{\textbf{Five Products Closely Related to the Proposal}}



\begin{enumerate} \renewcommand{\labelenumi}{[\arabic{enumi}]}

\item
S.-C.~T. Choi, Y.~Ding, F.~J. \HickernellFJ, L.~Jiang, {\relax Ll}.~A.
{Jim\'enez Rugama}, D.~Li, J.~Rathinavel, X.~Tong, K.~Zhang, Y.~Zhang, and
X.~Zhou, \emph{{GAIL}: {G}uaranteed {A}utomatic {I}ntegration {L}ibrary
	(versions 1.0--2.2)}, MATLAB software, \href{http://gailgithub.github.io/GAIL_Dev/} 
	{\nolinkurl{gailgithub.github.io/GAIL_Dev/}}, {2013--2017}.

\item S.-C.~T. Choi, Y.~Ding, F.~J. \HickernellFJ, and X.~Tong, \emph{Local adaption
	for approximation and minimization of univariate functions}, J. Complexity
\textbf{40} (2017), 17--33, {DOI} \\10.1016/j.jco.2016.11.005.

\item N.~Clancy, Y.~Ding, C.~Hamilton, F.~J. Hickernell, and Y.~Zhang, \emph{The cost
  of deterministic, adaptive, automatic algorithms: Cones, not balls}, J.
  Complexity \textbf{30} (2014), 21--45, {DOI} 10.1016/j.jco.2013.09.002.

%\item G.~E. Fasshauer, F.~J. Hickernell, and H.~Wo\'{z}niakowski, \emph{Average case
%  approximation: Convergence and tractability of {G}aussian kernels}, {M}onte
%  {C}arlo and Quasi-{M}onte {C}arlo Methods 2010 (L.~Plaskota and
%  H.~Wo\'{z}niakowski, eds.), Springer-Verlag, Berlin, 2012, pp.~329--344.

 
\item F.~J. \HickernellFJ, L.~Jiang, Y.~Liu, and A.~B. Owen, \emph{Guaranteed conservative fixed width confidence intervals via {M}onte {C}arlo sampling}, {M}onte {C}arlo and Quasi-{M}onte {C}arlo Methods 2012 (J.~Dick, F.~Y. Kuo,
G.~W. Peters, and I.~H. Sloan, eds.), Springer-Verlag, Berlin, 2014, pp.~105--128,
 {DOI} 10.1007/978-3-642-41095-6.

\item
F.~J. \HickernellFJ, {\relax Ll}.~A. {Jim\'enez Rugama}, and D.~Li,
\emph{Adaptive quasi-{M}onte {C}arlo methods for cubature}, Contemporary
Computational Mathematics --- a celebration of the 80th birthday of {I}an
{S}loan (J.~Dick, F.~Y. Kuo, and H.~Wo\'zniakowski, eds.), Springer-Verlag,
2018, pp.~597--619, {DOI} 10.1007/978-3-319-72456-0.


  
  

\end{enumerate}

\bigskip

\centerline{\textbf{Five Other Significant Products}}

\begin{enumerate} \renewcommand{\labelenumi}{[\arabic{enumi}]}
\setcounter{enumi}{5}
%\item G.~E. Fasshauer, F.~J. Hickernell, and H.~Wo\'{z}niakowski, \emph{On dimension-independent rates of convergence for function approximation with {G}aussian kernels}, SIAM J. Numer.\ Anal. \textbf{50} (2012), 247--271, {DOI} 10.1137/10080138X.

\item G.~E. Fasshauer, F.~J. Hickernell, and H.~Wo\'{z}niakowski, \emph{On   
dimension-independent rates of convergence for function approximation with {G}aussian kernels}, 
SIAM J. Numer.\ Anal. \textbf{50} (2012), 247--271, {DOI} 10.1137/10080138X.

\item F.~J. Hickernell, \emph{A generalized discrepancy and quadrature error bound},
  Math.\ Comp. \textbf{67} (1998), 299--322, {DOI}
  10.1090/S0025-5718-98-00894-1.
  
%\item
%F.~J. \HickernellFJ, H.~S. Hong, P.~L'{\'E}cuyer, and C.~Lemieux,
%  \emph{Extensible lattice sequences for quasi-{M}onte {C}arlo quadrature},
%  SIAM J. Sci.\ Comput. \textbf{22} (2000), 1117--1138, {DOI}
%  10.1137/S1064827599356638, {\em MR} 2001h:65032, {\em ZM} 974.65004.

\item F.~J. \HickernellFJ, \emph{The trio identity for quasi-{M}onte {C}arlo error
	analysis}, {M}onte {C}arlo and Quasi-{M}onte {C}arlo Methods: {MCQMC},
{S}tanford, USA, {A}ugust 2016 (P.~Glynn and A.~Owen, eds.), Springer
Proceedings in Mathematics and Statistics, Springer-Verlag, Berlin, 2018, pp.\ 3--27, 
{DOI} 10.1007/978-3-319-91436-7.
  
%\item \label{TOMACS} H.~S. Hong and F.~J. Hickernell, \emph{Algorithm 823: Implementing 
%scrambled digital nets}, ACM Trans.\ Math.\ Software \textbf{29} (2003), 95--109, {DOI} 
%10.1145/779359.779360.

\item F.~J. Hickernell and M.~Q. Liu, \emph{Uniform designs limit aliasing}, Biometrika \textbf{89} 
(2002), 893--904, {DOI} 10.1093/biomet/89.4.893.

\item B.~Niu, F.~J. Hickernell, T.~M\"uller-Gronbach, and K.~Ritter,
\emph{Deterministic multi-level algorithms for infinite-dimensional
	integration on {$\mathbb{R}^{\mathbb{N}}$}}, J. Complexity \textbf{27}
(2011), 331--351, {DOI} 10.1016/j.jco.2010.08.001.

\end{enumerate}

\medskip

\noindent {\large \textbf{Synergistic Activities}}

\begin{enumerate} \renewcommand{\labelenumi}{[\arabic{enumi}]}
	   
		\item \textbf{Mentoring.} My research has involved dozens of high school, BS, MS, MPhil and PhD students as 
		well as post-doctoral scholars.  My mentoring has included summer research experiences for 
		students funded by NSF and other sources.  Not only have 
		students gained experience discovering new mathematics, they have learned to 
		organize and communicate their discoveries in research group meetings, conference 
		presentations, and publications.  Students and post-docs that I have mentored have gone on to 
		further study, academic positions, and various positions in the commercial world.
		
		For over twelve years I was department chair at Illinois Tech before stepping down in 
		2017.  During that time I hired and encouraged eight new tenure track assistant professors: 
		three were women, six earned tenure and promotion to associate professor, and six 
		successfully 
		competed for external funding.  Two of the women hires, who have received tenure and been 
		promoted to associate professor and have received external funding, were in statistics, an area 
		where our department previously lacked strength.   
		
		In May, 2017, I was appointed director of a 
		newly 
		created Center for Interdisciplinary Scientific 
		Computation (CISC), and charged with raising the profile of our scientific computation research 
		and 
		education at Illinois Tech. CISC is connecting faculty from science, engineering, business, and 
		human sciences. 	In October 2018, I was appointed Vice Provost of Research for the university, and I passed the directorship of CISC to a new female faculty.

	    \item \textbf{Editorial work.} Associate Editor, {\em International Journal of Numerical Analysis and Modeling} 
	    (2003--2018), {\em Journal of Complexity} (1999--present), {\em Journal of Mathematical 
	    Research with Applications} (2010--present),
		{\em Mathematics of Computation}, 2008--2017, {\em SIAM Journal on Numerical Analysis}, 
		2005--2018.
		
%		\item Sixteen invited conference talks since 2004
	
		\item \textbf{Conference and program organization.} Organizer, Program Committee Member, and/or Steering Committee Member for a number of international conferences, including the {\em Third through \href{http://mcqmc2018.inria.fr}{Eleventh
		International Conferences on Monte Carlo and
		Quasi-Monte Carlo Methods in Scientific Computing}}, biennially 1998--present.  Program 
		Leader for the Statistical and Applied Mathematical Sciences Institute (SAMSI) 
	2017--18 
	\href{https://www.samsi.info/programs-and-activities/year-long-research-programs/2017-18-program-quasi-monte-carlo-high-dimensional-sampling-methods-applied-mathematics-qmc/
		}{\emph{Program on Quasi-Monte Carlo and High-Dimensional Sampling Methods for Applied 
		Mathematics (QMC)}}.  Organizer of a number of conference special sessions.

		
		
		\item \textbf{Software.} Worked with developers to promote the inclusion of software for generating low 
		discrepancy sequences in the \Matlab Statistics Toolbox, since R2008a, and in routine 
		G05YNF of the NAG library, since 2009.
		
			    \item \textbf{Awards.} Fellow of the Institute of Mathematical Statistics (elected 2007).  Recipient of the 
			    2016 
		Joseph F. Traub Prize for Achievement in Information-Based 
		Complexity.
		
		

		

\end{enumerate}


\end{document}

